\documentclass[a4paper,12pt]{article}
\usepackage{amssymb} % needed for math
\usepackage{amsmath} % needed for math
\usepackage[utf8]{inputenc} % this is needed for german umlauts
\usepackage[ngerman]{babel} % this is needed for german umlauts
\usepackage[T1]{fontenc}    % this is needed for correct output of umlauts in pdf
\usepackage[margin=2.5cm]{geometry} %layout
\usepackage{listings} % needed for the inclusion of source code

% the following is needed for syntax highlighting
\usepackage{color}

\definecolor{dkgreen}{rgb}{0,0.6,0}
\definecolor{gray}{rgb}{0.5,0.5,0.5}
\definecolor{mauve}{rgb}{0.58,0,0.82}

\lstset{ %
  language=Java,                  % the language of the code
  basicstyle=\footnotesize,       % the size of the fonts that are used for the code
  numbers=left,                   % where to put the line-numbers
  numberstyle=\tiny\color{gray},  % the style that is used for the line-numbers
  stepnumber=1,                   % the step between two line-numbers. If it's 1, each line
                                  % will be numbered
  numbersep=5pt,                  % how far the line-numbers are from the code
  backgroundcolor=\color{white},  % choose the background color. You must add \usepackage{color}
  showspaces=false,               % show spaces adding particular underscores
  showstringspaces=false,         % underline spaces within strings
  showtabs=false,                 % show tabs within strings adding particular underscores
  frame=single,                   % adds a frame around the code
  rulecolor=\color{black},        % if not set, the frame-color may be changed on line-breaks within not-black text (e.g. commens (green here))
  tabsize=4,                      % sets default tabsize to 2 spaces
  captionpos=b,                   % sets the caption-position to bottom
  breaklines=true,                % sets automatic line breaking
  breakatwhitespace=false,        % sets if automatic breaks should only happen at whitespace
  title=\lstname,                 % show the filename of files included with \lstinputlisting;
                                  % also try caption instead of title
  keywordstyle=\color{blue},          % keyword style
  commentstyle=\color{dkgreen},       % comment style
  stringstyle=\color{mauve},         % string literal style
  escapeinside={\%*}{*)},            % if you want to add a comment within your code
  morekeywords={*,...}               % if you want to add more keywords to the set
}

% this is needed for forms and links within the text
\usepackage{hyperref}
\usepackage[toc,page]{appendix}

\usepackage{tikz}
\usepackage{pgfplots}

%%%%%%%%%%%%%%%%%%%%%%%%%%%%%%%%%%%%%%%%%%%%%%%%%%%%%%%%%%%%%%%%%%%%%%
% Variablen                                                          %
%%%%%%%%%%%%%%%%%%%%%%%%%%%%%%%%%%%%%%%%%%%%%%%%%%%%%%%%%%%%%%%%%%%%%%
\newcommand{\authorName}{Ali Asgari Khoshouyeh (Student \#24868739)}
\newcommand{\tags}{\authorName, my, tags}
\title{CPEN 502 Assignment-a: Backpropagation (BP)}
\author{\authorName}
\date{\today}

%%%%%%%%%%%%%%%%%%%%%%%%%%%%%%%%%%%%%%%%%%%%%%%%%%%%%%%%%%%%%%%%%%%%%%
% PDF Meta information                                               %
%%%%%%%%%%%%%%%%%%%%%%%%%%%%%%%%%%%%%%%%%%%%%%%%%%%%%%%%%%%%%%%%%%%%%%
\hypersetup{
  pdfauthor   = {\authorName},
  pdfkeywords = {\tags},
  pdftitle    = {Backpropagation}
}

%%%%%%%%%%%%%%%%%%%%%%%%%%%%%%%%%%%%%%%%%%%%%%%%%%%%%%%%%%%%%%%%%%%%%%
% THE DOCUMENT BEGINS                                                %
%%%%%%%%%%%%%%%%%%%%%%%%%%%%%%%%%%%%%%%%%%%%%%%%%%%%%%%%%%%%%%%%%%%%%%
\begin{document}

\maketitle

\section{Simple BP and Binary Representation}
Set up your network in a 2-input, 4-hidden and 1-output configuration. Apply the XOR training set. Initialize weights to random values in the range -0.5 to +0.5 and set the learning rate to 0.2 with momentum at 0.0.

Define your XOR problem using a binary representation. Draw a graph of total error against number of epochs. On average, how many epochs does it take to reach a total error of less than 0.05? You should perform many trials to get your results, although you don’t need to plot them all.

\begin{center}

\begin{tikzpicture}
    \begin{axis}[
        xlabel=a) epochs,
        width=\textwidth,
        ylabel=MSE,
        xticklabel style={rotate=15},          
]
    \addplot[smooth,mark=.,blue] table {a.tex};
    \addlegendentry{Momentum = 0, Binary}
    \end{axis}
    \end{tikzpicture}

\end{center}

Out of \textbf{300} trials, on average it took \textbf{2538.2966666666666} epochs for the simple backpropagation algorithm to achieve the error of less than 0.05. 

Please note that through this report the backpropagation is done layer by layer from the last layer, i.e. first the last layer parameters are updated, then a new error signal is calculated for the hidden layer. 

\section{Bipolar representation}
This time use a bipolar representation. Again, graph your results to show the total error varying against number of epochs. On average, how many epochs to reach a total error of less than 0.05?


\begin{center}

\begin{tikzpicture}
    \begin{axis}[
        xlabel=b) epochs,
        width=\textwidth,
        ylabel=MSE]
    \addplot[smooth,mark=.,blue] table {b.tex};
    \addlegendentry{Momentum = 0, Bipolar}
    \end{axis}
    \end{tikzpicture}

\end{center}

Out of \textbf{300} trials, on average it took \textbf{271.5466666666667} epochs for the simple backpropagation algorithm to achieve the error of less than 0.05. 

\section{Adding momentum}
Now set the momentum to 0.9. What does the graph look like now and how fast can 0.05 be reached?


\begin{center}

\begin{tikzpicture}
    \begin{axis}[
        xlabel=c) epochs,
        width=\textwidth,
        ylabel=MSE]
    \addplot[smooth,mark=x,blue] table {c.tex};
    \addlegendentry{Momentum = 0.9, Bipolar}
    \end{axis}
    \end{tikzpicture}

\end{center}

Out of \textbf{300} trials, on average it took \textbf{25.756666666666668} epochs for the simple backpropagation algorithm to achieve the error of less than 0.05. 


\begin{appendices}
\section{Source Codes}
    \lstinputlisting[language=Java,caption=autograd/Addition.java]{../src/main/java/autograd/Addition.java}
\lstinputlisting[language=Java,caption=autograd/Exponentiation.java]{../src/main/java/autograd/Exponentiation.java}
\lstinputlisting[language=Java,caption=autograd/IInitializer.java]{../src/main/java/autograd/IInitializer.java}
\lstinputlisting[language=Java,caption=autograd/IOperator.java]{../src/main/java/autograd/IOperator.java}
\lstinputlisting[language=Java,caption=autograd/IVariable.java]{../src/main/java/autograd/IVariable.java}
\lstinputlisting[language=Java,caption=autograd/Multiplication.java]{../src/main/java/autograd/Multiplication.java}
\lstinputlisting[language=Java,caption=autograd/Negation.java]{../src/main/java/autograd/Negation.java}
\lstinputlisting[language=Java,caption=autograd/Operation.java]{../src/main/java/autograd/Operation.java}
\lstinputlisting[language=Java,caption=autograd/Operator.java]{../src/main/java/autograd/Operator.java}
\lstinputlisting[language=Java,caption=autograd/Parameter.java]{../src/main/java/autograd/Parameter.java}
\lstinputlisting[language=Java,caption=autograd/Sigmoid.java]{../src/main/java/autograd/Sigmoid.java}
\lstinputlisting[language=Java,caption=autograd/UniformInitializer.java]{../src/main/java/autograd/UniformInitializer.java}
\lstinputlisting[language=Java,caption=dataset/BinaryToBipolarWrapper.java]{../src/main/java/dataset/BinaryToBipolarWrapper.java}
\lstinputlisting[language=Java,caption=dataset/DataPoint.java]{../src/main/java/dataset/DataPoint.java}
\lstinputlisting[language=Java,caption=dataset/IDataSet.java]{../src/main/java/dataset/IDataSet.java}
\lstinputlisting[language=Java,caption=dataset/XORBinaryDataSet.java]{../src/main/java/dataset/XORBinaryDataSet.java}
\lstinputlisting[language=Java,caption=nn/BipolarSigmoid.java]{../src/main/java/nn/BipolarSigmoid.java}
\lstinputlisting[language=Java,caption=nn/ConvergenceCollector.java]{../src/main/java/nn/ConvergenceCollector.java}
\lstinputlisting[language=Java,caption=nn/Factory.java]{../src/main/java/nn/Factory.java}
\lstinputlisting[language=Java,caption=nn/IFitCallback.java]{../src/main/java/nn/IFitCallback.java}
\lstinputlisting[language=Java,caption=nn/ILayer.java]{../src/main/java/nn/ILayer.java}
\lstinputlisting[language=Java,caption=nn/Linear.java]{../src/main/java/nn/Linear.java}
\lstinputlisting[language=Java,caption=nn/MinimumSquaredError.java]{../src/main/java/nn/MinimumSquaredError.java}
\lstinputlisting[language=Java,caption=nn/Model.java]{../src/main/java/nn/Model.java}
\lstinputlisting[language=Java,caption=nn/Sigmoid.java]{../src/main/java/nn/Sigmoid.java}
\lstinputlisting[language=Java,caption=optimization/GradientDescent.java]{../src/main/java/optimization/GradientDescent.java}
\lstinputlisting[language=Java,caption=optimization/ILoss.java]{../src/main/java/optimization/ILoss.java}
\lstinputlisting[language=Java,caption=optimization/IOptimizer.java]{../src/main/java/optimization/IOptimizer.java}
\lstinputlisting[language=Java,caption=autograd/VariableTest.java]{../src/test/java/autograd/VariableTest.java}
\lstinputlisting[language=Java,caption=nn/NeuralNetworkTest.java]{../src/test/java/nn/NeuralNetworkTest.java}
\lstinputlisting[language=Java,caption=optimization/GradientDescentTest.java]{../src/test/java/optimization/GradientDescentTest.java}

\end{appendices}

\end{document}
